\documentclass[10pt,a4paper,oneside,titlepage]{report}

%%%%%%%%%%%%%%%%%%%%%%%%%%%%%%%%%%%%%%%%%%%%%%%%%%%%%%%%%%%%%%%%%%%%%%%%%%%%%%%%
%%%%%%%%%%%%%%%%%%%%%%%% PACKAGES %%%%%%%%%%%%%%%%%%%%%%%%%%%%%%%%%%%%%%%%%%%%%%
%%%%%%%%%%%%%%%%%%%%%%%%%%%%%%%%%%%%%%%%%%%%%%%%%%%%%%%%%%%%%%%%%%%%%%%%%%%%%%%%

\usepackage{times}
\usepackage[frenchb]{babel}
\usepackage{hyperref}
\hypersetup{pdfborder={0 0 0}, colorlinks=true, urlcolor=blue,
linkcolor = black, citecolor = black}
\usepackage[utf8]{inputenc}
\usepackage[T1]{fontenc}
\usepackage{amsmath}
\usepackage{amsfonts}
\usepackage{amscd}
\usepackage{amstext}
\usepackage{amssymb}
\usepackage{pifont}
\usepackage{xcolor}
\usepackage{multicol}
\usepackage{color}
\usepackage{mathrsfs}
\usepackage{graphicx}
\graphicspath{{pictures/}}
\usepackage{calligra}
\usepackage{amsthm}
\usepackage{multirow}
\usepackage{tabularx}
\usepackage{layout}
\usepackage{moreverb} %% boxedverbatim
\usepackage{xtab} %% xtabular
\usepackage{placeins} %% vider le cache de flottants
\usepackage{longtable} %% longtable

%%%%%%%%%%%%%%%%%%%%%%%%%%%%%%%%%%%%%%%%%%%%%%%%%%%%%%%%%%%%%%%%%%%%%%%%%%%%%%%%
%%%%%%%%%%%%%%%%%%%%%%%% COLORS %%%%%%%%%%%%%%%%%%%%%%%%%%%%%%%%%%%%%%%%%%%%%%%%
%%%%%%%%%%%%%%%%%%%%%%%%%%%%%%%%%%%%%%%%%%%%%%%%%%%%%%%%%%%%%%%%%%%%%%%%%%%%%%%%

\definecolor{darkred}{rgb}{0.85,0,0}
\definecolor{darkblue}{rgb}{0,0,0.7}
\definecolor{darkgreen}{rgb}{0,0.6,0}
\definecolor{darko}{rgb}{0.93,0.43,0}
\definecolor{quote}{rgb}{0.7,0.7,0.7}
\definecolor{maintitle}{rgb}{0.66,0,0.22}
\definecolor{title}{rgb}{0,0.5,0.5}
\definecolor{forestgreen}{rgb}{0.14,0.54,0.13}
\definecolor{cyan4}{rgb}{0,0.54,0.54}
\definecolor{firebrick4}{rgb}{0.54,0.1,0.1}
\definecolor{gris}{gray}{0.45}

%%%%%%%%%%%%%%%%%%%%%%%%%%%%%%%%%%%%%%%%%%%%%%%%%%%%%%%%%%%%%%%%%%%%%%%%%%%%%%%%
%%%%%%%%%%%%%%%%%%%%%%%% COLOR COMMANDS %%%%%%%%%%%%%%%%%%%%%%%%%%%%%%%%%%%%%%%%
%%%%%%%%%%%%%%%%%%%%%%%%%%%%%%%%%%%%%%%%%%%%%%%%%%%%%%%%%%%%%%%%%%%%%%%%%%%%%%%%

\newcommand{\forest}[1]{\textcolor{forestgreen}{#1}}
\newcommand{\cyan}[1]{\textcolor{cyan4}{#1}}
\newcommand{\firebrick}[1]{\textcolor{firebrick4}{#1}}
\newcommand{\maintitlecolor}[1]{\textcolor{maintitle}{#1}}
\newcommand{\titre}[1]{\textcolor{title}{#1}}
\newcommand{\dred}[1]{\textcolor{darkred}{\textbf{#1}}}
\newcommand{\dgre}[1]{\textcolor{darkgreen}{\textbf{#1}}}
\newcommand{\dblu}[1]{\textcolor{darkblue}{\textbf{#1}}}
\newcommand{\dora}[1]{\textcolor{darko}{\textbf{#1}}}
\newcommand{\gre}[1]{\textcolor{darkgreen}{#1}}
\newcommand{\blu}[1]{\textcolor{darkblue}{#1}}
\newcommand{\ora}[1]{\textcolor{darko}{#1}}
\newcommand{\rouge}[1]{\textcolor{darkred}{#1}}
\newcommand{\quotecolor}[1]{\textcolor{quote}{#1}}

%%%%%%%%%%%%%%%%%%%%%%%%%%%%%%%%%%%%%%%%%%%%%%%%%%%%%%%%%%%%%%%%%%%%%%%%%%%%%%%%
%%%%%%%%%%%%%%%%%%%%%%%%%%% THEMING %%%%%%%%%%%%%%%%%%%%%%%%%%%%%%%%%%%%%%%%%%%%
%%%%%%%%%%%%%%%%%%%%%%%%%%%%%%%%%%%%%%%%%%%%%%%%%%%%%%%%%%%%%%%%%%%%%%%%%%%%%%%%

\newcommand{\newterm}[1]{\textit{#1}}
\newcommand{\strong}[1]{\textbf{\titre{#1}}}
\newcommand{\desctitle}[1]{\underline{#1}}
%\newcommand{\iitem}{\item[$\blacktriangleright$]}
%\newcommand{\iiitem}{\item[$\bullet$]}

%%%%%%%%%%%%%%%%%%%%%%%%%%%%%%%%%%%%%%%%%%%%%%%%%%%%%%%%%%%%%%%%%%%%%%%%%%%%%%%%
%%%%%%%%%%%%%%%%%%%%%%%%% ENTETE %%%%%%%%%%%%%%%%%%%%%%%%%%%%%%%%%%%%%%%%%%%%%%%
%%%%%%%%%%%%%%%%%%%%%%%%%%%%%%%%%%%%%%%%%%%%%%%%%%%%%%%%%%%%%%%%%%%%%%%%%%%%%%%%

\begin{sffamily}

\title{
\begin{Huge}
\maintitlecolor{Gestion d'un centre de traitement de l'information}
\end{Huge}\\
\vspace*{1cm}
\begin{LARGE}\titre{\textit{\textbf{Résumé}}}\end{LARGE}
}


\author{
\vspace*{1cm} \\
\hbox{\raisebox{0.4em}{\vrule depth 2pt height 0.4pt width \textwidth}}
\vspace*{3cm} \\
\firebrick{\textbf{\begin{LARGE}Xavier Dubuc\end{LARGE}}} \\$ $\\
\textit{2\textsuperscript{ème} Master en Sciences Informatiques} \\
\textit{Finalité spécialisée} \\
(\url{xavier.dubuc@alumni.umons.ac.be}) \\
}


\end{sffamily}

%%%%%%%%%%%%%%%%%%%%%%%%%%%%%%%%%%%%%%%%%%%%%%%%%%%%%%%%%%%%%%%%%%%%%%%%%%%%%%%%
%%%%%%%%%%%%%%%%%%%%%%%% DOCUMENT %%%%%%%%%%%%%%%%%%%%%%%%%%%%%%%%%%%%%%%%%%%%%%
%%%%%%%%%%%%%%%%%%%%%%%%%%%%%%%%%%%%%%%%%%%%%%%%%%%%%%%%%%%%%%%%%%%%%%%%%%%%%%%%

\addtolength{\hoffset}{-2cm}
\addtolength{\textwidth}{3cm}

% Fix section numbering.
\renewcommand*\thesection{\arabic{section}}

\begin{document}

\maketitle

\newpage

\textbf{Contributeurs:}
\renewcommand{\labelitemi}{$\bullet$}

\begin{itemize}
\item 2016
\begin{itemize}
\item Julien Delplanque (\url{julien.delplanque@student.umons.ac.be})
\end{itemize}
\end{itemize}

\newpage

\tableofcontents

\newpage

\section{Objectifs du cours}

Formuler des recommandations afin d'atteindre des objectifs en terme :

\begin{itemize}
\item de satisfaction des utilisateurs;
\item de qualité des services offerts;
\item de rentabilité des projets;
\item et d'écologie.
\end{itemize}

\section{Le centre informatique}

\subsection{Ses missions}

\begin{enumerate}
\item \textbf{\strong{Mise en place, exploitation et maintenance} des ressources
informatiques et audiovisuelles pour les infrastructures communes}\\
	$\hookrightarrow$ \newterm{fournir à l'ensemble des acteurs de l'université
    (étudiants, enseignants, chercheurs, \\
	\indent $\quad$ services administratifs, direction) des :
	\begin{itemize}
		\item[$\qquad\blacktriangleright$] ressources informatiques
        performantes, sécurisées et disponibles,
		\item[$\qquad\blacktriangleright$] moyens audiovisuels
	\end{itemize}
	\indent $\quad$ sur ses différents campus.} \\
	\indent $\longrightarrow$ accéder aux bibliothèques, à son compte, Moodle,
    aux salles infos, ... \\

\item  \textbf{\strong{Prévision de l'évolution des systèmes informatiques et
élaboration de la politique informatique} (2 à 5 ans) en collaboration avec un
conseil de direction} \\
$\hookrightarrow$ \textit{objectifs de développement à moyen et à long termes
(schéma directeur, plan stratégique)}

Le \underline{schéma directeur} est un plan d'évolution de l'environnement
informatique de l'entreprise à terme de 2 à 5 ans. Il est composé de :
\begin{enumerate}
\item analyse de l’existant: ``État des lieux''
\item identification des besoins : objectifs stratégiques et opérationnels
\end{enumerate}
Il implique \textbf{la direction générale} (choix stratégiques),
\textbf{le service informatique} (missions), \textbf{les utilisateurs}
(attentes), le \textbf{comité de direction} (approbation des grands axes,
projets, investissements) et les \textbf{Comités fonctionnels} (aspects
pratiques). Il doit être ``adaptable'' (faire face aux imprévus : changement de
technologie, retard,...)\\

\item \strong{Adaptation des décisions en fonction des évolutions
technologiques.} \\

\item \strong{Fourniture de services centraux d'intérêt commun}
\textit{(messageries,...).}\\

\item \strong{Architecture, développement, évolution des infrastructures de
réseau informatique et suivi des incidents.}\\

\item \strong{Développement, suivi et maintenance des applications de gestion
spécifiques et des équipements associés.}\\

\item \strong{Développement et maintenance du site web.}\\

\item \strong{Mise en œuvre d'une politique de sécurité, respect de la charte de
bon usage des ressources informatiques, déploiement d'outils, maintien de la
sécurité et suivi des incidents.}\\

\item \strong{Mise en place de mesures pour assurer la sécurité des données des
utilisateurs face aux virus, tentatives de piratage, spam et autres dangers.} \\

\item \strong{Formation, conseil, aide et assistances des/aux utilisateurs.} \\

\item \strong{Reconditionnement (recyclage) du matériel.} \\

\item \strong{Installation et mise à disposition d'équipements audiovisuels et
multimédia.} \\

\item \strong{Relations avec les fournisseurs, suivi de contrats de maintenance.
} \\

\item \strong{Vitrine informatique.} \\

\item \strong{Veille technologique.} \\

\item \strong{Mise en \oe uvre de moyens permettant de contrôler la situation
par rapport aux objectifs définis et facilitant les prises de décision.} \\
$\hookrightarrow$ \newterm{Par exemple : tableau de bord, audits internes, ...
} \\
Par \underline{tableau de bord}, on entend réellement un tableau de bord comme
celui d'une voiture. Il s'agit là d'un outil permettant de visualiser l'état de
fonctionnement des services. Ce fonctionnement est évalué en fonction
d'objectifs fixés en se basant sur des indicateurs de performances. C'est devenu
un outil de pilotage et d'aide à la décision.\\

Par \underline{Indicateur}, on entend ``un élément ou un ensemble d’éléments
d’information représentative par rapport à une préoccupation ou un objectif,
résultant de la mesure tangible ou de l’observation d’un état, de la
manifestation d’un phénomène, d’une réalisation''\footnote{\url{http://formation
.enap.ca/tbord/intro_frame.htm}}. Celui-ci est principalement formulé en terme
de:
\begin{itemize}
\item Quantité: Nombre, taux, etc\dots;
\item Qualité: Valeur relative;
\item Montant: Coût, revenus;
\item Temps: Fréquence, délai.
\end{itemize}
\end{enumerate}

\subsection{Gestion administrative et financière}

\subsubsection{Gestion financière}

Le budget doit couvrir les coûts de matériel, logiciel (achat, licence),
maintenance, équipements spéciaux, redevances télécoms, fournitures,
consommables, formations, frais généraux, frais de personnel,\dots
Le C.I. évalue ce budget et soumet une proposition chiffrée à la direction en
vue de l'acceptation du budget pour l'année suivante. Une facturation interne
contribue à l'amortissement des équipements divers (5 cents la photocopie par
exemple).

\subsubsection{Gestion administrative}

Obligations légales (archivage), contrats d'assurance, \dots et suivi des
prestations.

\subsubsection{Gestion du personnel}

Former régulièrement le personnel aux nouvelles technologie afin d'augmenter la
motivation et améliorer la productivité.

\subsection{Composition et organigramme}

\noindent A l'UMons, le département informatique se compose de $2$ services :
\begin{enumerate}
\item \textbf{\strong{le Centre d'Informatique} (\firebrick{CI})}:
\newterm{personnel dont les compétences couvrent les domaines de l'informatique
et des réseaux, de l'audiovisuel et du multimédia, de la téléphonie.}
\item \textbf{\strong{le Centre d'Informatique Administrative}
(\firebrick{CIA})}: \newterm{analyse, développement, projets}
\end{enumerate}

Ce département relève d'un \strong{Conseil de l'Informatique} composé de
différents représentants de l'Université.\\

En entreprise, cet organigramme possède une composition variable suivant la
taille de celle-ci.

\subsection{Les métiers}

\begin{itemize}
\item La direction informatique,
\item un secrétariat rattaché à la direction,
\item un service études/développement,
\item un service système,
\item un service exploitation,
\item un service méthodologie,
\item un administrateur de bases de données (DBA),
\item un ingénieur en informatique décisionnelle,
\item un concepteur et administrateur Web,
\item un spécialiste en sécurité informatique,
\item un ingénieur réseau,
\item un gestionnaire de parc microinformatique,
\item veille technologique,
\item \dots
\end{itemize}

\subsubsection{La direction informatique}
Celle-ci est responsable du bon du fonctionnement, de l'évolution et de la
sécurité des systèmes d'information de l'entreprise. Elle préconise des
solutions techniques adaptées à la stratégie de l’entreprise et assure la
cohérence des solutions mises en place et le respect des normes établies.\\

D'autre part, elle élabore le plan informatique et gère le budget du centre et
encadre le personnel du service informatique et coordonne les activités des
membres de son service.\\

La direction assure aussi les négociations et les discussions avec la direction
générale et les autres services de l’entreprise et établit et entretient les
contacts avec les sociétés extérieures (fournisseurs, sociétés prestataires de
services, \dots).\\

Enfin, très souvent, elle est sollicitée par divers interlocuteurs, le directeur
informatique doit faire preuve d’une grande souplesse et d’une grande
disponibilité.\\

Dépendant de la taille de l’entreprise et du développement de ses systèmes
d’information, on parlera de responsable informatique, de directeur
informatique, de directeur des systèmes d’information\dots\\

En résumé la direction doit faire preuve des qualité suivante:
\begin{itemize}
\item dirigeant, manager et stratège;
\item gestionnaire;
\item technicien à l'esprit curieux.
\end{itemize}

\subsubsection{Le service développement/études}
Celui-ci analyse des besoins des utilisateurs et développe des applications
informatiques. Il est compétant en informatiques et en applications afin de bien
cerner les besoins de l’entreprise et se compose d'un responsable de service,
des chefs de projets et des informaticiens (analystes, programmeurs).\\

Détails des différents métiers:
\begin{itemize}
\item Le responsable de service: Il dirige le service, est responsable des
applications, participe à l’analyse des besoins des utilisateurs et évalue les
nouveaux projets en terme de rentabilité financière et d’avantage concurrentiel
pour l’entreprise. Il organise des réunions régulières avec les utilisateurs
finaux afin d'observer l'état d’avancement. Il planifie et organise des
contrôles réguliers des projets. Finalement, il assure la coordination et
la formation des membres de son équipe.
\item Le chef de projet: Il pilote les différentes phases de mise au point d’un
projet en animant une équipe d’analyste et de programmeurs. Il veille aussi à ce
que les logiciels s’intègrent dans le système existant. D'autre part, il a la
responsabilité de fixer les dates de début et fin de projet et organiser les
réunions de contrôle. Il doit avoir le goût de la communication et apprécier le
travail en équipe et idéalement il doit être un chef de projet informatique et
un chef de projet applicatif.
\item L'analyste fonctionnel: Il définit les besoins applicatifs
(méthode d’analyse) et établit les diagrammes de circulation de l’information.
Il réalise aussi la description logique des fichiers d’entrée et de sortie et
prépare un dossier pour l’analyste organique.
\item L'analyste organique: Il fait une proposition technique sur base de
l’analyse fonctionnelle : décrit les solutions matérielles et logicielles dans
un dossier technique et prépare le dossier de programmation : caractéristiques
des fichiers (format, taille, type d’enregistrement), format des écrans, \dots
Il doit aussi concevoir des programmes de tests, participer aux essais et au
lancement des applications, rédiger des modes d'emploi pour les utilisateurs
et assurer la maintenance des applications.\\
\end{itemize}

\textbf{Quelques recommandations:}
\begin{itemize}
\item Lors de la mise en place d'une application informatique, il faut veiller
à: utiliser un groupe d'utilisateurs pilote, fournir un suivi attentif, fournir
de la documentation de qualité, procéder au changement via une phase de
transition entre l'ancienne et la nouvelle application, prévoir une procédure
de reprise de l'existant et prévoir une possibilité de retour en arrière en cas
de problème.
\item Pour contrôler une application, il faut: évaluer régulièrement la qualité
de l’application, prévoir des procédures de reprise et de sauvegarde et veiller
à la satisfaction des utilisateurs.
\item Pour maintenir une application existante, il faut: fournir un accès aux
fichiers et aux programmes réglementés, documenter chaque modification et si
la maintenance devient trop lourde, envisager une réécriture de l'application.
\end{itemize}

\subsubsection{Le service système}
Ce service a comme responsabilité l'installation, le maintient et la
personnalisation des logiciels systèmes. Éventuellement, il installe les
logiciels développés par le service Études/développement. Il suit les offres
des constructeurs et l'évolution des logiciels. D'autre part, il participe
au plan de sécurité et assure le suivi des performance.\\

Le plan de backup consiste en la sauvegarde régulière des données et de la liste
des applications utilisée. Le plan de survie est destiné à assurer le maintien
des activités critiques de l'entreprise après un sinistre (en quelques heures).
Le plan de reprise est composé de procédures de reprise précises et détaillées
pour chaque activité de l'entreprise. Il permet le redémarrage de l’ensemble des
activités du centre.\\

Détails des différents métiers:
\begin{itemize}
\item Le responsable de service système: Il planifie l’installation des
logiciels en concertation avec les service Etudes et exploitation tout en
s'assurant que les utilisateurs sont satisfaits des services fournis. Il
contrôle les projets de son équipe (faisabilité technique, respect des
plannings, suivi financier) et assure une coordination technique.
\item L'ingénieur système: Il conçoit l’architecture du système informatique et
suit l’évolution des produits. Éventuellement, il propose les changements de
version de logiciels si celles-ci sont requises.
\item Le programmeur système: Il installe les produits livrés par les
fournisseurs et le service Études. d'autre part, il répertorie les problèmes
(via les logs, journaux, dumps, etc\dots) et applique les correctifs requis.
\end{itemize}

\subsubsection{Le service exploitation}
Ce service gère et organise le fonctionnement du centre au jour le jour. Il
organisation et planning des travaux, gère la salle et entretiens des relations
avec les fournisseurs, les prestataires de services et les utilisateurs.\\

Détails des différents métiers:
\begin{itemize}
\item Le responsable du service: Il gère l’organisation du travail, détermine
les procédures d’exploitation et optimise la production. D'autre part, il
répartit les tâches entre les membres de son équipe et organise le déroulement
des travaux. Il travaille en étroite collaboration avec les services des études
et système.
\item Chef de salle: Il contrôle l’environnement physique de la salle et
encadre les techniciens d’exploitation.
\item Technicien d’exploitation: Il démarre les systèmes, exécutent les
applications en fonction des plannings établis et surveille le fonctionnement
des équipements et sont à l’affût des incidents. C'est aussi lui qui lance les
impressions volumineuses, les sauvegardes, les traitements en classe batch en
fonction des contrats établis. D'autre part, il gère les stocks de papier et
diverses fournitures.
\end{itemize}

\subsubsection{Le service méthodologie}
Il assure le respect des méthodes de travail et définit les standards de
l'entreprise (plan directeur, choix des méthodes d’analyse, convention de
programmation,...)\\

Si tout le monde respecte les standards et les règles en vigueur, le
développement et la maintenance des applications seront facilités.

\subsubsection{Administrateur de bases de données}
L’administrateur de base de données a en charge la gestion et la protection des
bases de données de l’entreprise. Il conçoit la structure des bases de données.
Il en garantit la cohérence et la sécurité (confidentialité, sécurité,
intégrité). Il travaille en étroite collaboration avec les utilisateurs et les
autres services du centre. Son rôle devient de plus en plus complexe avec
l’évolution des systèmes informatiques (architectures décentralisées, avènement
des entrepôts de données (data warehouse), ouverture des systèmes d’information
vers le monde extérieur,\dots). Il doit bien entendu avoir une très bonne
connaissance du fonctionnement de l’entreprise et des ses applications.

\subsubsection{Ingénieur en informatique décisionnelle}
C'est un métier que l’on va rencontrer dans les entreprises de grandes tailles
et qui est en relation avec le souci croissant d’utiliser les données de manière
stratégique (mieux connaître l’entreprise, faciliter les prises de décision,
\dots). Il est chargé de la mise en œuvre et de la gestion du data warehouse
(c.f. cours de Datamining \& Datawarehousing). Il choisit les modèles de données
et les outils statistiques, analytiques et de présentation les plus pertinents.
Il doit avoir une très bonne connaissance du fonctionnement de l’entreprise et
des ses applications.

\subsubsection{Concepteur et administrateur Web}
Son rôle est variable dépendant de la taille de l'entreprise. Il peut s'agir
d'une personne qui travaille dans l’entreprise ou d'un consultant. Sur base
d’une analyse des besoins des utilisateurs, il prend en charge la conception du
site (cahier des charges, charte graphique, architecture technique, organisation
du contenu, aspects marketing, référencement, mise en exploitation, \dots). Il a
un esprit créatif, le sens de l’organisation et aime travailler en équipe.
Il doit maîtriser de nombreux langages et logiciels graphiques

\subsubsection{Spécialiste en sécurité informatique}
Dans certaines entreprises soit très grandes soit ou les données sont d’une très
grande importance stratégique on fait appel à un spécialiste de la sécurité
informatique. Dans les PME, et plus petites entreprises, cette fonction est
souvent déléguée à l’ensemble des intervenants du service informatique.
Le responsable en sécurité est chargé d’étudier et de mettre en œuvre des
mesures de protection des systèmes et des réseaux (audit, politique de sécurité,
anti-virus, firewall\dots). Il doit travailler en très étroite collaboration
avec les autres services de l’entreprise et avoir un bon sens de l’écoute et des
contacts.

\subsubsection{Ingénieur réseau}
Il assure la mise en place, la gestion et le développement des réseaux de
communication de l’entreprise. Il met tout en œuvre pour en assurer la
disponibilité. Il définit l’architecture des réseaux et étudie les solutions les
mieux adaptées compte tenu des besoins de l’entreprise (network designer -
architecte réseau). Ces tâches sont parfois sous-traitées à l’extérieur.
Il joue un rôle de conseiller dans le cadre des projets utilisant les réseaux en
veillant au bon dimensionnement des infrastructures réseau en fonction des
besoins des applications. Avec l’arrivée d’Internet et l’ouverture de
l’entreprise sur le monde extérieur de nombreux problèmes de sécurité sont
apparus (firewall, journaux, logiciel de détection d’intrusion, listes d’accès,
\dots).

\subsubsection{Gestionnaire de parc microinformatique}
Il est chargé de l’installation, de la gestion et de la maintenance du parc
micro-informatique. Il procède à l’inventaire des logiciels et des matériels.
Il joue un rôle d’assistance auprès des utilisateurs au niveau de la bureautique
et en cas de problème lié à l’utilisation du poste de travail.

\section{Conception d'une salle informatique}

\subsection{Recommandations du CLUSIF}

Les centres informatiques ont évolué. Ainsi dans les années 70 il s'agissait de
centres vitrines liés au prestige, dans les années 80 ce sont devenus des
centres \newterm{blockhaus} liés à la sécurité et dans les années 90 ils sont
entrés dans une phase d'intégration où ils sont composés d'une salle
informatique centrale, de serveurs délocalisés, de réseaux, \dots
\strong{Le problème majeur est d'envisager la sécurité d'un point de vue
global.} \\

Un centre informatique peut voir le jour suite à plusieurs situations : soit on
ouvre un nouveau centre (fermeture de l'ancien à cause d'un sinistre, fin de
bail, vétusté, \dots) soit on améliore un ancien centre (suite à une extension
ou une restructuration de l'entreprise). \\

\noindent Les acteurs d'un projet d'un centre informatique sont :
\begin{itemize}
\item \strong{la direction générale} qui possède le pouvoir de décision et qui
représente le maître d'\oe uvre. Cette maîtrise est souvent déléguée à un cadre
ou à une société extérieure,
\item \strong{la fonction informatique} qui doit impérativement \^etre associée
au projet car c'est elle qui doit produire le cahier des charges. \\
\end{itemize}

\noindent Avant l'implantation du centre, il faut suivre des \strong{étapes
préliminaires} :
\begin{itemize}
\item \strong{étude critique de la situation existante} : \textit{relever les
problèmes à tous les niveaux}\\(différents services, organisation générale,
effectifs,...),
\item \strong{analyse des différents flux} \textit{(physiques, humains,
logiques)},
\item \strong{analyse des besoins futurs en ressources humaines}
\textit{(formations, recrutement,...)}
\end{itemize}

Il faut bien étudier les fonctions du centre (\textit{héberger les serveurs, le
matériel de télécom, le personnel, ...}).

\subsubsection{Implantation du centre}

Il faut prendre en compte l'évolution et la possibilité d'extension au niveau
des \strong{aspects fonctionnels} \textit{(extension du personnel, missions,
...} et des \strong{aspects organisationnels} \textit{(encombrement, contraintes
liées au matériel,...)}. Il faut également prendre en compte les contraintes
géographiques en relevant les caractéristiques du site d'implantation (risques
naturels, risques du voisinage, ...). \\

Au niveau des besoins généraux, il faut utiliser une bonne nomenclature des
locaux-zones fonctionnelles et prendre en compte
\begin{enumerate}
\item \strong{les contraintes techniques}\\
il convient de faire un inventaire des matériels hébergés
\textit{(encombrement, poids, dégagement, climatisation, alimentation
électrique, ...) et du personnel \textit{(localisation, éclairage,...)}}.
$\Rightarrow$ \strong{premier bilan des besoins en terme de superficie,
d'alimentation électrique et thermique.}
\item \strong{les contraintes logistiques}\\
\textit{proximité des secours, chemins d'accès (personnel, livraison, ...),
stationnement, ...}
\item \strong{les contraintes organisationnelles}\\
\textit{disposition optimale des locaux}
\item \strong{les contraintes humaines}\\
\textit{installation et déménagement, proximité des moyens de transport,
possibilité de restauration}
\item \strong{les contraintes de planification}\\
\textit{date de début des travaux, de livraison du matériel, de mise en service,
...}
\item \strong{les contraintes financières}\\
\textit{coûts des différents postes (achat terrain ou immeuble, frais de
déménagement éventuels, ...)}
\end{enumerate}

\subsubsection{Objectifs de sécurité}

Définir ces objectifs
\begin{itemize}
\item pour les performances et conditions d'exploitation : \textit{plages de
fonctionnement, taux d'indisponibilité acceptable, condition d'intervention,
...},
\item pour l'entretien et la maintenance : \textit{a qui confier ces missions ?,
politique de redondance et de secours}.
\end{itemize}

\subsubsection{Menaces et parades}

\begin{itemize}
\item \strong{Dégâts des eaux}\\
\textbf{Parades} : Choix de l'implantation, système de drainage, conduites
apparentes, surélévation du matériel critique, ...
\item \strong{Dégâts du feu}\\
\textbf{Parades} : Mise en place de systèmes de détection et de protection,
maintenance rigoureuse, ...\\
\textbf{Mesures organisationnelles} : interdiction stricte de fumer, procédure
d'évacuation testée, ...
\item \strong{Coupures électriques}\\
\textbf{Parades} : UPS, groupe électrogène, ...
\item \strong{Défaut de climatisation}\\
\textbf{Parade} : mise en place d'un système redondant
\item \strong{Incidents de télécommunication}\\
\textbf{Parade} : dédoublement des lignes et du matériel critique
\item ...
\end{itemize}

\subsection{Implantation physique}

\subsubsection{Emplacement}

Définir judicieusement l'emplacement physique compte tenu de diverses
contraintes :
\begin{itemize}
\item \strong{sécurité des lieux} : éviter les zones trop isolées, minimum
d'ouvertures vers l'extérieur, inondations, foudre, ... ;
\item \strong{facilité d'accès} : utilisateurs, fournisseurs ;
\item \strong{contraintes de poids, d'encombrement} du matériel : surcharges
nécessaire du béton en salle et en salle de stockage ;
\item \strong{contraintes électriques et électromagnétiques}
\end{itemize}

\subsubsection{Architecture, dimensionnement}

Faire appels à des sociétés de service et aux constructeurs pour les normes
d'électricité et les techniques de sécurité incendie.\\
Divers éléments entrent en ligne de compte. La conception générale : une salle
pour les processeurs et les unités disques, une salle pour les imprimantes et
les robots de bandes magnétiques, une salle pour stocker les fournitures
(papier, ...), une aire de travail et une aire de repos et il faut prévoir des
possibilités d'extension. La hauteur conseillée d'une salle est de 3m et cette
dernière devrait être constituée d'un plancher technique (30 à 50cm de haut)
et d'un double plafond (30 à 50cm de haut) pour le passage des câbles, des
gaines techniques, de la climatisation, la ventilation, ... La surface du sol
dépendra de la place physique occupée par l'ordinateur et ses périphériques. Il
faut également prévoir une surface de dégagement autour de chaque dispositif
afin de faciliter l'accès lors des interventions sur le matériel et pour la
circulation du personnel du service d'exploitation.

\subsection{Techniques spéciales}

\subsubsection{Alimentation électrique de la salle}

Ses caractéristiques sont fournies par le constructeur. On recommande une ligne
d'alimentation séparée pour l'informatique. Tous les appareils de la salle
doivent être connectés à la même terre. Si une alimentation sans coupure est
requise, on a recours à un onduleur, un groupe no-break un \strong{UPS}
\textit{(Uninterruptible Power Supply)}. En cas de coupure, on a recours à un
générateur de courant comme un groupe électrogène diesel.\\

Un \textbf{UPS} permet :\begin{itemize}
\item une protection contre la foudre et les surtensions,
\item une régulation de tension,
\item une alimentation continue (uninterruptible),
\item une alimentation sinusoïdale de qualité informatique,
\item en mode normal, une conversion du courant secteur en un courant régulier
et de qualité informatique,
\item en cas d'absence ou de dégradation du courant secteur, génère une
alimentation sinusoïdale.
\end{itemize}

\subsubsection{Climatisation}

On calcule le dégagement de chaleur (dissipation) par effet Joule (+/- 500
Watt/$m^2$) produit par les appareils. En règle générale, les tolérances sont :
\begin{itemize}
\item \textbf{un taux d'humidité de l'air de 45 à 55\%}
\item \textbf{une température de 22 +/- 2$^\circ$C.}
\end{itemize}

\subsubsection{Détection et extinction incendie}

Supprimer les causes potentielles (ne pas fumer par exemple), ajouter des
détecteurs de fumée ou thermiques. Il existe des moyens appropriés de défense:
extinction avec du gaz carbonique ou de l'azote (avec avertissement pour la
sécurité des personnes). Il est bien entendu qu'il faut s'assurer que les locaux
sont vides avant d'utiliser de telles solutions sur base d'un plan d'évacuation.
Une autre technique : les sprinklers (les extincteurs automatiques à eau
accrochés au plafond).

\subsubsection{Autres}

\begin{itemize}
\item le c\^ablage réseau,
\item la détection d'eau dans le plancher technique, détection d'humidité,
système d'évacuation des eaux,
\item éclairage de secours,
\item contr\^ole d'accès,
\item surveillance et détection d'intrusion.
\end{itemize}

\subsection{Evaluation des besoins et exploitation des ressources}

\subsubsection{Etat des lieux - Analyse de l'existant}

Par exemple, lors du renouvellement des installations informatiques centrales de
l'université.
\begin{itemize}
\item relevé des applications,
\item nombre d'applications tournant en simultané,
\item nombre d'utilisateur connectés,
\item besoin en communication,
\item nombre de travaux batch,
\item satisfaction des utilisateurs, celui-ci est une fonction du temps de
réponse qui doit en général être $\leq$ à 1 seconde. Ce dernier est une
fonction de la charge qui elle même est une fonction des $n$ applications et
des $m$ utilisateurs. Il faut faire attention aux pics, et donc faire une
répartition uniforme des travaux.
\end{itemize}

Pour amener à bien cet état des lieux, il faudra également relever la
\strong{charge des processeurs} pour évaluer le besoin en puissance de CPU, le
\strong{taux d'utilisation des mémoires} pour évaluer le besoin en taille de la
mémoire et les \strong{accès disques} pour évaluer la capacité et les
performances de ces disques.

\subsubsection{Eléments à définir - Prise en compte des besoins}

Il faudra définir et prendre en compte :
\begin{itemize}
\item \textbf{les types d'applications} : \textit{commerciales, scientifiques,
bureautiques, ...}
\item \textbf{les caractéristiques de l'unité centrale} et
\textbf{des périphériques} \textit{(CPU, mémoire, canaux)} et se poser les
bonnes questions :
\begin{itemize}
\item quelles fonctions seront assignées au serveur ? (application, courrier,
imprimante, ...)
\item combien d'utilisateurs vont faire appel à ce serveur simultanément ?
\end{itemize}
\item \textbf{le besoin en support de données},
\item \textbf{les moyens de communication},
\item \textbf{les possibilités d'extensions futures},
\end{itemize}

\subsubsection{A - Disponibilité d'un système}

Le taux de disponibilité d'un système est donné par : $$\frac{MTBF}{MTBF+MTTR}$$
où $MTBF$ est la moyenne du temps de bon fonctionnement et $MTTR$ est la moyenne
du temps de toutes les réparations.

\subsubsection{B - Moyens pour réduire le MTTR}

\begin{itemize}
\item redondance matérielle,
\item stockage et redondance des données (RAID,SAN,...),
\item administration, suivi des incidents
\item contr\^ole de l'alimentation électrique (UPS).
\end{itemize}

\subsubsection{Qu'est-ce qui va entrer en compte dans la performance d'une
machine ?}

Cette performance n'est pas seulement due aux caractéristiques du processeur,
elle dépend également :
\begin{itemize}
\item du nombre d'unités de traitement,
\item du cycle machine,
\item de la capacité des mémoires,
\item du débit des bus reliant les mémoires et les processeurs,
\item des connexions mémoires centrales et auxiliaires,
\item du temps de réponse des différents étages de la pyramide des mémoires,
\item de la vitesse des I/O,
\item ...
\end{itemize}

\subsubsection{Comment va-t-on mesurer les performances ?}

Plusieurs coefficients : \begin{itemize}
\item le \textbf{MIPS} (\textit{millions d'instructions par seconde}),
uniquement valable au sein d'une m\^eme famille de machines,
\item le \textbf{MFLOPS} (\textit{Million of Floating Point Operations per
Second}),
\item le \textbf{LINPACK} basé sur un ensemble de programmes qui relèvent de
l'algèbre linéaire,
\item les indices \textbf{SPEC} (System Performance Evaluation Corporation) qui
est une association de constructeurs, universités, organisation de recherches,
consultants, ... qui utilise une suite de programmes (benchmarks) qui vont
permettre de prédire les performances d'une machine pour une charge de travail
déterminée. Ces programmes testent les performances du processeur, de la mémoire
et du compilateur.
\item \textbf{TPC} (avec des bases de données)
\end{itemize}

\subsubsection{Haute densité, haute disponibilité, une nouvelle génération de
data centers}

Nouvelles contraintes dues à la forte consommation et du fort dégagement de
chaleur de ces nouveaux data centers. Qui dit haute disponibilité, dit haute
disponibilité électrique qui est son critère principal. Les hébergeurs proposent
des bâtiments adaptés via différentes formules allant de $99,99 \%$ de taux de
service à $99,9999999 \%$.

\subsubsection{Hébergement}

Différentes solutions :
\begin{enumerate}
\item Externalisation complète ou en partie du centre informatique (fait par
les grands groupes comme Google),
\item hébergement de certaines ressources comme un site web,
\item site de back-up (sur un autre site de l'entreprise ou via un site miroir
pour les sites très sensibles ou encore un back-up dégradé sur un autre site de
l'entreprise).
\end{enumerate}

\section{Gestion du réseau}
La gestion du réseau est une tâche très complexe par les éléments qui le
constituent et par la gestion de l’entreprise et de l’informatique dans leur
ensemble (au cœur des activités de l’entreprise). L'arrêt du réseau représente
une perte financière, perte d’ ``image'' de marque. La gestion du réseau doit
être ``transparente'' (pendant celle-ci: temps de réponse correct, ressources
disponibles, qualité des services).\\

La politique de gestion du réseau doit s’inscrire dans le cadre d’une vision à
long terme (anticipation des besoins, être à l’écoute des nouveautés
technologiques), doit être crédible du point de vue des utilisateurs (leur
montrer qu’elle repose sur le souhait de répondre à leurs besoins) et du point
de vue de la direction (bonne maîtrise des coûts) et doit définir les
responsabilités des personnes impliquées dans la politique de gestion du réseau.
\\

La mise en place d’un réseau: méthodologie de cycle de vie du développement dont
les principales phases sont:
\begin{enumerate}
\item phase de compréhension des besoins des utilisateurs
\item critères de choix
\item choix des fournisseurs
\end{enumerate}
Une bonne évaluation des besoins et d’un dimensionnement optimal des
installations est importante. Il faut tenter de réduire les coûts informatiques
tout en améliorant la qualité du service rendu aux utilisateurs. Pour le bien,
une gestion externalisé ou infogérance doit être possible.\\

La gestion du parc informatique permet de mieux gérer et contrôler les coûts
(stratégie d’achat) et de réaliser certains bénéfices grâce à l’homogénéisation
(recensement exhaustif possible des actifs informatiques de l’entreprise,
logiciels de gestion d’inventaire).\\

Les fonctions d’un système de gestion de parc informatique s'inscrivent dans une
logique de qualité. Elles doivent s'interfacer avec les autres services de
gestion du réseau (gestion des performances, des incidents, de la comptabilité)
et de la gestion des ressources informatiques. La mise en place d’une politique
cohérente de remplacement des machines donne lieu à une meilleure anticipation
des besoins.\\

Pour la gestion de la qualité du service réseau, des critères de qualité
mesurables existent: disponibilité, accessibilité et temps de réponse,
fiabilité du réseau. La qualité de service d’un réseau est directement mesurable
à la satisfaction des utilisateurs (sondages). Une politique de qualité de
service nécessite la mise en place d’une structure organisationnelle adéquate.
Une observation suivie de la qualité de service permet de mesurer l’efficacité
du réseau (meilleure connaissance du trafic, des problèmes potentiels, des
tendances des utilisateurs). La prévision de l’évolution des besoins des
utilisateurs et du réseau est facilitée.

\section{La sécurité}
Les entreprises sont dépendantes de leur système d’information mais ceux-ci sont
vulnérables. Il est nécessaire de protéger les systèmes en contrôlant: la
disponibilité et la performance du matériel et des logiciels, en protégeant les
accès aux systèmes, en protégeant les données et en développant une confiance
dans l'identité des correspondant. D'autre part, une prise de conscience et une
sensibilisation est nécessaire et cela doit faire partie de la politique de
sécurité de l'entreprise.\\

On attend d'un système d'information qu'il soit disponible (ressources
matérielles et logicielles) via de la redondance et des procédures de
sauvegarde, de reprise. On attend aussi une confidentialité et une intégrité
des données via des processus de contrôle (authentification) et du chiffrement.
Il faut qu'une stratégie globale de sécurisation (protection, détection,
réaction) soit mise en place.\\

Il faut envisager la sécurité de façon globale via une approche ``top-down'';
trouver le bon compromis entre les mesures de sécurité, la performance des
systèmes et le coût des mesures; informer les utilisateurs des risques encourus,
des procédures à respecter, des éventuelles sanctions (assurer le respect de la
charte de sécurité) et responsabiliser les personnes qui manipulent des données
sensibles.\\

\noindent
\textbf{Définition 1}:\emph{Une politique de sécurité doit déterminer les objets
à sécuriser; elle identifie les menaces à prendre en compte; elle définit le
périmètre de sécurité; elle spécifie l’ensemble des lois, règlements et
pratiques qui régissent la façon de gérer, protéger et diffuser les informations
et autres ressources sensibles au sein d’un système spécifique, d’une entité.}\\

\noindent
\textbf{Définition 2}:\emph{Une politique de sécurité est un ensemble de règles
établies sur base d’une analyse de risques. Le but est de minimiser la survenue
des risques (mesures préventives) et faire face à leur éventuelle apparition
(mesures correctives) afin de diminuer le risque informationnel de
l’entreprise.}

\subsection{Recommandation générale}
Il faut, tout d'abord, nommer un responsable au sein du service IT. Ensuite,
il faut élaborer la politique de sécurité en collaboration avec la direction,
le personnel technique, le personnel juridique et les utilisateurs. Il convient
d'avoir une vision globale et de mettre en œuvre des mesures raisonnables.
La politique doit être: applicable, compréhensible et explicite.\\

Une analyse de risque est requise pour déterminer les mesures de sécurité
appropriées pour l’entreprise en fonction des risques encourus. La méthode est
la suivante:
\begin{enumerate}
\item Identifier les informations sensibles de l’entreprise et on les classe en
fonction de leur sensibilité.
\item Procéder au recensement de tous les éléments qui constituent le système
d’information où qui interagissent avec lui (matériel, infrastructure réseau,
applications et utilisateurs).
\item Analyse des menaces, des vulnérabilités et des risques (identifier les
``menaces'' qui pèsent sur l’entreprise, identifier les ``vulnérabilités''
et évaluer les conséquences si un risque était avéré).
\item Décider le niveau de risque acceptable par l’entreprise et le niveau de
protection à accorder aux données.
\item Mettre en place des solutions pour minimiser les risques identifiés
(mesures préventives) et pour assurer la reprise en cas de sinistre (mesures
correctives, plan de reprise).
\item Il faudra vérifier que les mesures de sécurité ne nuisent pas trop aux
performances des systèmes auquel cas il faut revoir la politique de sécurité.
\item Auditer la solution de sécurité.
\end{enumerate}

\subsection{Mise en œuvre pratique}
Différents modèles ou normes existent:
\begin{itemize}
\item BS 7799/ISO 17799
\item CobiT
\item EBIOS
\item ITIL
\item Les Critères Communs (CC)/ ISO 15408
\item MARION et MEHARI
\item OCTAVE
\item Clusif
\item \dots
\end{itemize}

\end{document}
